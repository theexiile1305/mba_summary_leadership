\documentclass{cheatsheet}
\title{Leadership}
\IfFileExists{revision.tex}{\include{revision.tex}}{}
\begin{document}
    \section{Führung im Überblick}
    \begin{sectionbox}{Bedeutung guter Führung}
        \begin{itemize}
            \item historisch \& aktuell von großer Bedeutung\\
            \ra durch Mächtigkeit \& Dynamik von Persönlichkeit
            \item kaum beschreibbar, wie man richtig führt
            \item unterschiedliche Ansätze \& Erfolgsfaktoren
            \item mehrere Variablen zur Messung des Erfolgs
        \end{itemize}
    \end{sectionbox}
    \begin{warningbox}{Ansätze der Führung}
        \begin{itemize}
            \item \textbf{Rolle}: Verknüpft mit Erwartungen an Führungskraft
            \item \textbf{Soziale Einflussnahme}: Einwirken auf Mitarbeitenden durch Führung, um sinnvolles Verhalten zu erreichen
        \end{itemize}
    \end{warningbox}
    \begin{sectionbox}{Wege der Einflussnahme}
        \begin{itemize}
            \item Ziele \& Maßnahmen der Organisation
            \item Motivation der Mitarbeiter zur Zielerreichung
            \item Vertrauen \& Kooperation zwischen Mitarbeitern
            \item Ressourcenallokation für Ziele \& Maßnahmen
            \item Ausgestaltung von Bürokratie \& Systemen
        \end{itemize}
    \end{sectionbox}
    \begin{hintbox}{Indirekter Einfluss auf Unternehmenserfolg}
        \begin{itemize}
            \item Vertriebler mit/ohne motivierte Mitarbeiter
            \item Fehlerquote in Produktion durch Sensibilisierungen 
        \end{itemize}
    \end{hintbox}
    \subsection{Betrachtungen von Managementdenkern}
    \begin{normbox}{\subsubsection{Konzept \emph{Management by objectives} von \emph{Drucker}}}
        \begin{itemize}
            \item Mitarbeitende müssen geführt werden
            \item Resultate anstatt Beliebtheit im Vordergrund
            \item Sichtbarkeit als Führungskraft mit gutem Beispiel
            \item Führungskraft macht Verantwortung aus
        \end{itemize}
    \end{normbox}
    \begin{normbox}{\subsubsection{\emph{John P. Kotter}, Prof. Havard Businuess School}}
        \begin{itemize}
            \item Wei soll Zukunft gestaltet werden?
            \item Koordination von Mitarbeitern \& Zielen
            \item Inspiration von Mitarbeitenden\\
            \ra Erreichen von Zielen trotz Hindernisse
        \end{itemize}
    \end{normbox}
    \begin{normbox}{\subsubsection{\emph{Jack Welsh}, ehemaliger Vorstand General Electric}}
        Abbildung Führungsübernahme als Prozess
        \begin{itemize}
            \item eigene Wachstum zur Vorbereitung auf Führung
            \item Verantwortungsübernahme
            \item Wachstum Anderer im Mittelpunkt 
        \end{itemize}
    \end{normbox}
    \begin{normbox}{\subsubsection{\emph{von Rosenstel}, deutschsprachiger Raum}}
        \begin{itemize}
            \item zielbezogene Einflussnahme\\
            \ra \textbf{Strukturell}: Führung durch Strukturen\\
            \ra \textbf{Personal}: Führung durch Personen
            \item sodass Geführte
            \item bestimmte Ziele erreichen
            \item abgeleitet aus Zielen des Unternehmens
        \end{itemize}
    \end{normbox}
    \begin{warningbox}{Was ist Leadership?}
        Komplexer sozialer Prozess als unmittelbare Kommunikation zwischen Führungskraft \& Mitarbeitenden
    \end{warningbox}
    \subsection{Kriterien des Führungserfolgs}
    \begin{normbox}{\subsubsection{Allgemeines}}
        \begin{itemize}
            \item 3 Ebenen zur Beurteilung des Führungserfolgs\\
            \ra Dyade (FK \& MA), Gruppe/Team \& Organisation
            \item Interaktion mittels Kommunikation
            \item Soziale Führungsbeziehung\\
            \ra wechselseitiges aufeinander bezogenes Handeln
            \item Unmittelbarkeit: direkt, persönlich \& individuell
            \item Intentionalität
            \item Verhaltensbeeinflussung
            \item Akzeptanz
        \end{itemize}
    \end{normbox}
    \begin{normbox}{\subsubsection{Effizienzkriterien}}
        \begin{itemize}
            \item ökonomische Effizienz: typische Erfolgszahlen
            \item Leistungsprozesseffizienz\\
            \ra materielle \& immaterielle Leistungsprozesse
            \item Personeneffizenz\\
            \ra arbeitsbezogener \& individuelle Einstellungen
        \end{itemize}
    \end{normbox}
    \begin{normbox}{\subsubsection{Effektivitätskriterien}}
        \begin{itemize}
            \item Leistung: Qualität, Zeitersparnis \& kreative Leistung
             \item Kohäsion: Teamzusammenhalt \& Frustrationstoleranz
             \item Zufriedenheit: Arbeitszufriedenheit \& Commitment
        \end{itemize}
    \end{normbox}
    \begin{normbox}{\subsubsection{Stakeholderansatz}}
        \begin{itemize}
            \item höchstmöglicher Zufriedenheit der Stakeholder
        \end{itemize}
    \end{normbox}
    \begin{warningbox}{Was beinhaltet gute Führung?}
        Messen \& Beurteilen von Vorgesetzten aufgrund von Kriterien anhand unternehmerischer Entscheidungen 
    \end{warningbox}
    \section{Führungstheorien im Wandel der Zeit}   
        \begin{sectionbox}{Eigenschaftstheorie - \emph{Great Man Theory}}
        \begin{itemize}
            \item Sozialdarwinistisches Elitedenken
            \item Geborene Führungspersönlichkeiten\\
            \ra Keine Trainier- oder Erlernbarkeit
            \item Ablehnung der Entscheidungspartizipation von MA
            \item ursprüngliche Eigenschaften\\
            \ra zeitstabil \& situationsabhängig\\
            \ra messbar \& klar feststellbar
            \item statisch ohne Berücksichtigung der Führungssituation
            \item Weiterentwicklung via Stogdill \& \emph{BIG FIVE}
        \end{itemize}
    \end{sectionbox}
        \begin{hintbox}{Stogdill (1948): Attribute für Führungserfolg}
            Fähigkeit, Leistung, Verwantwortung, Partizipation, Status
        \end{hintbox}
        \begin{hintbox}{Persönlichkeitsskala \emph{BIG FIVE}}
            Extraversion, Verträglichkeit, Offenheit, Gewissenhaftigkeit und emotionale Stabilität
        \end{hintbox}
    \begin{sectionbox}{Führungsstil und -person}
        \begin{itemize}
            \item 
        \end{itemize}
    \end{sectionbox}
    \begin{sectionbox}{Berücksichtigung der Situation}
        \begin{itemize}
            \item 
        \end{itemize}
    \end{sectionbox}
    \begin{sectionbox}{Systematische Führung}
        \begin{itemize}
            \item 
        \end{itemize}
    \end{sectionbox}
        \begin{sectionbox}{Symbolische Führung}
        \begin{itemize}
            \item 
        \end{itemize}
    \end{sectionbox}
    \begin{sectionbox}{Transaktionale und transformationale Führung}
        \begin{itemize}
            \item 
        \end{itemize}
    \end{sectionbox}
    \begin{sectionbox}{Führung im Spannungsfeld}
        \begin{itemize}
            \item 
        \end{itemize}
    \end{sectionbox}
    \section{Neue Leadership-Ansätze}
    \section{Belastungen, Work-Life-Balance und Selbstmanagement}
    \section{Motivation, Kommunikation und Beurteilung}
    \section{Teams}
    \section{Aktuelle Trends und Debatten}
\end{document}