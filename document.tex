\documentclass{cheatsheet}
\title{Leadership}
\IfFileExists{revision.tex}{\include{revision.tex}}{}
\begin{document}
    \section{Führung im Überblick}
    \begin{sectionbox}{Bedeutung guter Führung}
        \begin{itemize}
            \item indirekter \& wesentlicher Einfluss auf Unternehmen
            \item verschiedene Messvariablen für Führungserfolg\\
            \ra mittelbarer Bezug auf Erfolgsfaktoren
            \item Assoziation als Rolle oder sozialer Einflussprozess
        \end{itemize}
        \begin{hintbox}{Beeinflussung von Führungskraften}
            \begin{itemize}
                \item Ziele \& Maßnahmen von Organisationseinheiten
                \item  Motivation von Mitarbeitenden Ziele zu erreichen
                \item Vertrauen \& Kooperation unter Mitarbeitenden
                \item Ressourcenallokation für Ziele \& Maßnahmen
                \item Ausgestaltung von Bürokratie \& Systemen
            \end{itemize}
        \end{hintbox}
        \begin{warningbox}{Beantwortung von Fragen von guter Führung}
            \begin{itemize}
                \item Selektion: Persönlichkeitseigenschaften
                \item Modifikation: Strategien, um erfolgreich zu führen
            \end{itemize}
        \end{warningbox}    
    \end{sectionbox}
    \begin{sectionbox}{Führung \& Leadership: Begriffsdefinition}
        \begin{warningbox}{Definition}
            Zielbezogene Beeinflussung von Unterstellten durch Vorgesetzte mithilfe von Kommunikationsmitteln.
        \end{warningbox}
        \begin{hintbox}{Einflussnahme von Mitarbeitenden über die Führung}
            via \emph{Strukturen} und \emph{Personen}
        \end{hintbox}
    \end{sectionbox}
    \begin{sectionbox}{Kriterien des Führungserfolgs}
        \textbf{Beziehung auf drei Ebenen}: \emph{Dyade} (bilaterale Beziehungsebene), Gruppe/Team, Organisation\\
        \ra \emph{gute Führung} kann zum Unternehmenserfolg führen\\
        \textbf{Beurteilung durch drei Effizienzkriterien}
        \begin{itemize}
            \item ökonomische Effizienz: typische Erfolgszahlen\\
            \ra Gewinn, Rentabilität oder Umsatz\\
            \item Leistungsprozesseffizienz\\
            \ra materielle Leistungsprozesse: Ausschuss, Unfälle\\
            \ra immaterielle Leistungsprozesse: Problemlösung\\
            \item Personeneffizenz\\
            \ra arbeitsbezogener Einstellungen: Zufriedenheit\\
            \ra individuelle Einstellungen: Einflussakzeptanz
        \end{itemize}
         \textbf{Beurteilung durch drei Effektivitätskriterien}
         \begin{itemize}
             \item Leistung: Qualität, Zeitersparnis, kreative Leistung
             \item Kohäsion: Teamzusammenhalt, Frustrationstoleranz
             \item Zufriedenheit: Arbeitszufriedenheit, Commitment
         \end{itemize}
         \textbf{höchstmögliche Zufriedenheit bei vielen Stakeholdern}
    \end{sectionbox}
\end{document}