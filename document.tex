\documentclass{cheatsheet}
\title{Leadership}
\IfFileExists{revision.tex}{\include{revision.tex}}{}
\begin{document}
    \section{Führung im Überblick}
    \begin{sectionbox}{Bedeutung guter Führung}
        \begin{itemize}
            \item historisch \& aktuell von großer Bedeutung\\
            \ra durch Mächtigkeit \& Dynamik von Persönlichkeit
            \item kaum beschreibbar, wie man richtig führt
            \item unterschiedliche Ansätze \& Erfolgsfaktoren
            \item mehrere Variablen zur Messung des Erfolgs
        \end{itemize}
    \end{sectionbox}
    \begin{warningbox}{Ansätze der Führung}
        \begin{itemize}
            \item \textbf{Rolle}: Verknüpft mit Erwartungen an FK
            \item \textbf{Soziale Einflussnahme}: Einwirken auf Mitarbeitenden durch Führung, um sinnvolles Verhalten zu erreichen
        \end{itemize}
    \end{warningbox}
    \begin{sectionbox}{Wege der Einflussnahme}
        \begin{itemize}
            \item Ziele \& Maßnahmen der Organisation
            \item Motivation der MA zur Zielerreichung
            \item Vertrauen \& Kooperation zwischen MA
            \item Ressourcenallokation für Ziele \& Maßnahmen
            \item Ausgestaltung von Bürokratie \& Systemen
        \end{itemize}
    \end{sectionbox}
    \begin{hintbox}{Indirekter Einfluss auf Unternehmenserfolg}
        \begin{itemize}
            \item Vertriebler mit/ohne motivierte MA
            \item Fehlerquote in Produktion durch Sensibilisierungen 
        \end{itemize}
    \end{hintbox}
    \subsection{Betrachtungen von Managementdenkern}
    \begin{normbox}{\subsubsection{Konzept \emph{Management by objectives} von \emph{Drucker}}}
        \begin{itemize}
            \item Mitarbeitende müssen geführt werden
            \item Resultate anstatt Beliebtheit im Vordergrund
            \item Sichtbarkeit als FK mit gutem Beispiel
            \item FK macht Verantwortung aus
        \end{itemize}
    \end{normbox}
    \begin{normbox}{\subsubsection{\emph{John P. Kotter}, Prof. Havard Businuess School}}
        \begin{itemize}
            \item Wei soll Zukunft gestaltet werden?
            \item Koordination von MA \& Zielen
            \item Inspiration von Mitarbeitenden\\
            \ra Erreichen von Zielen trotz Hindernisse
        \end{itemize}
    \end{normbox}
    \begin{normbox}{\subsubsection{\emph{Jack Welsh}, ehemaliger Vorstand General Electric}}
        Abbildung Führungsübernahme als Prozess
        \begin{itemize}
            \item eigene Wachstum zur Vorbereitung auf Führung
            \item Verantwortungsübernahme
            \item Wachstum Anderer im Mittelpunkt 
        \end{itemize}
    \end{normbox}
    \begin{normbox}{\subsubsection{\emph{von Rosenstel}, deutschsprachiger Raum}}
        \begin{itemize}
            \item zielbezogene Einflussnahme\\
            \ra \textbf{Strukturell}: Führung durch Strukturen\\
            \ra \textbf{Personal}: Führung durch Personen
            \item sodass Geführte
            \item bestimmte Ziele erreichen
            \item abgeleitet aus Zielen des Unternehmens
        \end{itemize}
    \end{normbox}
    \begin{warningbox}{Was ist Leadership?}
        Komplexer sozialer Prozess als unmittelbare Kommunikation zwischen FK \& Mitarbeitenden
    \end{warningbox}
    \subsection{Kriterien des Führungserfolgs}
    \begin{normbox}{\subsubsection{Allgemeines}}
        \begin{itemize}
            \item 3 Ebenen zur Beurteilung des Führungserfolgs\\
            \ra Dyade (FK \& MA), Gruppe/Team \& Organisation
            \item Interaktion mittels Kommunikation
            \item Soziale Führungsbeziehung\\
            \ra wechselseitiges aufeinander bezogenes Handeln
            \item Unmittelbarkeit: direkt, persönlich \& individuell
            \item Intentionalität
            \item Verhaltensbeeinflussung
            \item Akzeptanz
        \end{itemize}
    \end{normbox}
    \begin{normbox}{\subsubsection{Effizienzkriterien}}
        \begin{itemize}
            \item ökonomische Effizienz: typische Erfolgszahlen
            \item Leistungsprozesseffizienz\\
            \ra materielle \& immaterielle Leistungsprozesse
            \item Personeneffizenz\\
            \ra arbeitsbezogener \& individuelle Einstellungen
        \end{itemize}
    \end{normbox}
    \begin{normbox}{\subsubsection{Effektivitätskriterien}}
        \begin{itemize}
            \item Leistung: Qualität, Zeitersparnis \& kreative Leistung
             \item Kohäsion: Teamzusammenhalt \& Frustrationstoleranz
             \item Zufriedenheit: Arbeitszufriedenheit \& Commitment
        \end{itemize}
    \end{normbox}
    \begin{normbox}{\subsubsection{Stakeholderansatz}}
        \begin{itemize}
            \item höchstmöglicher Zufriedenheit der Stakeholder
        \end{itemize}
    \end{normbox}
    \begin{warningbox}{Was beinhaltet gute Führung?}
        Messen \& Beurteilen von Vorgesetzten aufgrund von Kriterien anhand unternehmerischer Entscheidungen 
    \end{warningbox}
    \section{Führungstheorien im Wandel der Zeit}   
        \begin{sectionbox}{Eigenschaftstheorie - \emph{Great Man Theory}}
        \begin{itemize}
            \item Sozialdarwinistisches Elitedenken
            \item Geborene Führungspersönlichkeiten\\
            \ra Keine Trainier- oder Erlernbarkeit
            \item Ablehnung der Entscheidungspartizipation von MA
            \item ursprüngliche Eigenschaften\\
            \ra zeitstabil \& situationsabhängig\\
            \ra messbar \& klar feststellbar
            \item statisch ohne Berücksichtigung der Führungssituation
            \item Weiterentwicklung via Stogdill \& \emph{BIG FIVE}
        \end{itemize}
    \end{sectionbox}
        \begin{hintbox}{Stogdill (1948): Attribute für Führungserfolg}
            Fähigkeit, Leistung, Verantwortung, Partizipation, Status
        \end{hintbox}
        \begin{hintbox}{Persönlichkeitsskala \emph{BIG FIVE}}
            Extraversion, Verträglichkeit, Offenheit, Gewissenhaftigkeit und emotionale Stabilität
        \end{hintbox}
    \begin{sectionbox}{Führungsstilkontinuum}
        \begin{itemize}
            \item Keine Berücksichtigung von Mitarbeitenden/Situation
            \item eindimensional von autoritär bis teil-autonom
            \item Richtiger Führungsstil in Abhängigkeit von\\
            \ra Qualifikation Mitarbeitende \& Arbeitsinhalt\\
            \ra Persönlichkeitsmerkmale FK \& Teamgröße\\
            \ra Akzeptanz FK im Team \& Branche
        \end{itemize}
    \end{sectionbox}
    \subsection{Berücksichtigung der Situation}
    \begin{normbox}{\subsubsection{Kontingenztheorie nach \emph{Fiedler}}}
        \begin{itemize}
            \item Führungsstil in Abhängigkeit von Situation
            \item Führungssituation bedingt durch
            \begin{itemize}
                \item \emph{Führungskraft-Mitarbeiter-Beziehung}\\
                \ra Wertschätzung der FK durch MA
                \item \emph{Positionsmacht}\\
                \ra Einfluss der FK infolge hierarchischer Stellung
                \item \emph{Strukturierungsgrad der Aufgabe}\\
                \ra Anzahl wiederkehrender Elemente, deren Planbarkeit \& Überprüfbarkeit der Leistungen\\
                \ra Je klarer die Aufgabe strukturiert, desto leichter fallen Koordination \& Kontrolle der MA
            \end{itemize}
            \item Erfolg dieser Führungsstile bedingt je nach Situationen\\
            \ra günstig \& ungünstig: leistungsbezogene Führung\\
            \ra mittelgünstig: mitarbeiterbezogene Führung
            \item Kritik: Laborexperiment \rightleftharpoons !Proof in Feldstudien
        \end{itemize}
    \end{normbox}
    \begin{normbox}{\subsubsection{Reifegradmodell nach \emph{Hersey \& Blanchard}}}
        \begin{itemize}
            \item Führungsverhalten abhängig von Reifegrad des MA\\
            \ra psychologisch: Motivation \& Verantwortung\\
            \ra funktional: Ausbildung \& Erfahrung
            \ra individualisierte Führung \ra kein Gießkannenprinzip
            \begin{hintbox}{$psychologische \ Reife \bullet funktionale \ Reife$}
                \ra unterweisend: gering \& gering\\
                \ra verkaufend: hoch \& gering\\
                \ra partizipierend: gering \& hoch\\
                \ra delegierend: hoch \& hoch
            \end{hintbox}
            \item Ziel: Maximierung des Reifegrads des MA
            \item Plausibilität des Modells, hohe Praxistauglichkeit 
            \item Kritik: !empirischer Beleg
        \end{itemize}
    \end{normbox}
    \begin{normbox}{\subsubsection{Entscheidungsmodell von \emph{Vroom \& Yetton}}}
        \begin{itemize}
            \item empirisch nachgewiesener Entscheidungsbaum (6Q)
            \item Entscheidungen des Führungsstil basierend auf\\
            \ra Autokratie\\
            \ra konsultativ auf Beratung der Geführten\\
            \ra Gruppe
            \item Führungsentscheidung basierend auf\\
            \ra Qualität \& Akzeptanz der Entscheidung\\
            \ra Ökonomie des Entscheidungsverhaltens
        \end{itemize}
    \end{normbox}
    \begin{sectionbox}{Systematische Führung}
        \begin{itemize}
            \item Führungshandlung in komplex vernetzten Sozialgefüge mit vielen direkten \& indirekten Reaktionen\\
            \ra Einflussnahme auf System durch Kommunikation\\
            \ra Begreifen komplexr und Bilateraler Interaktionen
            \item Systeme: alle Relationen \& Interaktionen zueinander
            \item System: Getriebe \& Unternehmen
            \item Basiselement: Kommunikation
            \item Problem: FK ist Teil des System \rightleftharpoons Einflüsse!
        \end{itemize}
    \end{sectionbox}
        \begin{sectionbox}{Symbolische Führung}
        \begin{itemize}
            \item Symbolisierung von Werten \& Überzeugungen\\
            \ra oft große Interpretation von MA
            \item aufgesetzte Symbolik häufig unauthentisch
            \item unterschiedliche Interpretation der Führungshandlung
            \item Kritik: empirisch ungenügend erforscht
        \end{itemize}
    \end{sectionbox}
    \subsection{Transaktionale \& transformationale Führung}
    \begin{normbox}{\subsubsection{Allgemeines}}
        \begin{itemize}
            \item Verbreitung seit 30y des Ansatz in Industrie/Science
            \item transaktionaler Führung \ra transformationale Führung\\
            \ra positive Verstärkung \& Vorbildfunkton der FK
        \end{itemize}
    \end{normbox}
    \begin{normbox}{\subsubsection{Transaktionale Führung}}
        \begin{itemize}
            \item Basis: lerntheoretisches Prinzip der Verstärkung
            \ra operante Konditionierung \rightleftharpoons \emph{carrots-and-sticks}
            \item FK kontrolliert Aufgaben \& Ziele des MA
            \item korrespondiert mit \emph{Management by Exception}
            \item Fairer Austausch zwischen FK \& MA
        \end{itemize}
    \end{normbox}
    \begin{normbox}{\subsubsection{Transformationale Führung}}
        \begin{itemize}
            \item Motivation der MA durch Werte \& Gefühle
            \item Ergänzung transaktionalen Führung durch\\
            \ra langfristige \& übergeordnete Werte ergänzen kurzfristige materielle Ziele
            \item fairer Austausch von Leistung \rightleftharpoons Gegenleistung
            \item Fokus: Vertrauen, Sinn, Inspiration \& Wertschätzung
        \end{itemize}
    \end{normbox}
    \begin{warningbox}{Motivation der MA durch Werte \& Gefühle auf 4 Ebenen}
        \begin{itemize}
            \item \emph{Inspirational Motivation}: überzeugende Vermittlung attraktiver Visionen
            \item \emph{Idealized Influence}: glaubwürdige Verkörperung eines Vorbildes
            \item \emph{Intellectual Stimulation}: Anregung zum unabhängigen Denken, Unterstützung von Veränderung
            \item \emph{Individualized Consideration}: durch die Unterstützung der MA-Entwicklung, individuelle Wertschätzung
        \end{itemize}
    \end{warningbox}
    \begin{hintbox}{Rollenverhalten von Führungskräften - Gegenüberstellung}
        \begin{itemize}
            \item Koordinationsmechanismen\\
            \ra Verträge, Belohnung \& Bestrafung (\emph{TA})\\
            \ra Begeisterung \& Vertrauen, Kreativität (\emph{TF})
            \item MA-Motivation: extrinsisch (\emph{TA}), intrinsisch (\emph{TF})
            \item Zielerreichung: kurzfristig (\emph{TA}), langfristig (\emph{TF})
            \item Zielinhalte: materiell (\emph{TA}), ideel (\emph{TF})
            \item Rolle der FK: Instrukteur (\emph{TA}), Lehrer \& Coach (\emph{TF})
        \end{itemize}
    \end{hintbox}
    \begin{sectionbox}{Führungstheorien im Wandel der Zeit: \\ Führung im Spannungsfeld}
        \begin{itemize}
            \item Keine richtige Führungstheorie\\
            \ra da Abhängigkeit von zu vielen Kontextfaktoren\\
            \ra Erweiterung des Handlungsspektrums einer FK\\
            \ra Handeln mit Ziel Wertebalance zu finden
        \end{itemize}
        \begin{hintbox}{Was sind Werte?}
            \begin{itemize}
                \item Perspektiven \& Sichtweisen von Menschen
                \item \emph{Wertewandel}: Menschen wandeln sich\\
                \ra entdecken \& schaffen neue Werte
                \item zu verschiedenen Zeiten stehen unterschiedliche Werte im Mittelpunkt
                \item erheben Sollanspruch, jedoch nicht Realisierung
            \end{itemize}
        \end{hintbox}
        \begin{itemize}
            \item Management \ra Resultate \& messbare Ziele\\
            \ra Führung = Leiten durch Werte
            \item Werte: gemeinsames Grund- \& Einverstädnis
            \item Werte: Handlungsrahmen bei Zielerreichung
            \item 3 Ebenen: materiell, sozial \& geistig
        \end{itemize}
    \end{sectionbox}
    \section{Neue Leadership-Ansätze}
    \section{Belastungen, Work-Life-Balance und Selbstmanagement}
    \section{Motivation, Kommunikation und Beurteilung}
    \section{Teams}
    \section{Aktuelle Trends und Debatten}
\end{document}